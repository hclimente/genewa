\documentclass{article}

\usepackage{amsmath,epsfig,bm, amssymb,graphicx,algorithm,algorithmic,color,
  bbm}
\usepackage{hyperref}

\begin{document}

\title{Network-based GWAS on familial breast cancer: method comparison and new biomarkers}
\author{H\'ector Climente-Gonz\'alez, Christine Lonjou, Chlo\'e-Agathe Azencott}

\maketitle

\section{Methods}

\subsection{GENESIS}

GENESIS is a GWAS study of familial breast cancer on the French population. The
index cases are patients with a breast cancer affected sister and no BRCA1/2
mutations \cite{sinilnikova_genesis:_2016}. The controls are unaffected
colleagues and/or friends of the cases. In total, the GENESIS dataset consists
of 2\,566 samples, of which 1\,293 are cases and 1\,273 are controls. The genotyping
platform was iCOGS, a custom Illumina array designed
to study genetic susceptibility of hormone-related cancers \cite{sakoda_turning_2013}. It contains ~211\,155
SNPs previously associated with cancer susceptibility, survival, or other
cancer-related traits; and in selected genes and pathways.

We used the age of the patients as covariate. Specifically, the
age at diagnosis for cases, and the age at the interview for controls.

\subsection{Preprocessing and quality control}

A PCA revealed no differential population structure between cases and controls.

28 samples with ?? were removed. A subset of 20 duplicated SNPs in FGFR2 were also removed.
Following the suggested protocol in ??, we LD pruned the dataset with $R^2$ 0.75.

We imputed missing genotypes using IMPUTE2 \cite{howie_flexible_2009}. The reference panel were the
genotypes from the 1000 Genomes Project Phase 3 \cite{the_1000_genomes_project_consortium_global_2015}. We selected 503 samples, the
number of samples of European origin in the reference panel. After imputation, we discarded
SNPs with a minor allele frequency lower than 0.1\%, not in Hardy - Weinberg
equilibrium (P-value \textless 0.001), and/or those with missing values on more than 10\% of the samples. In addition, we removed the samples with more than 10\% missing genotypes. The final dataset included 1\,279 controls and 1\,298 cases, genotyped over 128\,595 SNPs. 
\subsection{SNP- and gene-level association scores}


\subsection{High-weight module discovery algorithms}

\subsubsection{Node scores}

We used PLINK v1.90 \cite{chang_second-generation_2015} to fit a per-SNP
logistic regression, using the age as covariate. The assumed model of
inheritance was additive. Then, we computed a P-value of the null hypothesis of
the coefficient being different than 0. We used VEGAS2v2 to compute the
gene-level association score \cite{mishra_vegas2:_2015}. In order to map a SNP
to a gene we relied on their physical position, being within 50 kb of the
borders of the gene.

\subsubsection{Network}

We used the physical interactions from BioGRID version 3.5.171
\cite{oughtred_biogrid_2019} to get protein-protein interactions when required.
Out of the six methods tested, five use a protein-protein interaction network.

\subsubsection{Methods used}

Finding the highest-scoring, densely interconnected subnetwork on a graph is an
open problem in the field. Hence, several solutions have been proposed to the
problem. In this paper, we apply 5 methods designed to explore the
protein-protein interaction network, and one method, SConES, which explores
SNP-networks.

\noindent {\bf Heinz} \cite{dittrich_identifying_2008}

\noindent {\bf Hierarchichal HotNet} \cite{reyna_hierarchical_2018}

\noindent {\bf jActiveModules}

\noindent {\bf LEAN} \cite{gwinner_network-based_2016}

\noindent {\bf SConES} \cite{azencott_efficient_2013}

\noindent {\bf SigMod}

\subsection{Validation of selected biomarkers}

\subsubsection{Classification accuracy of selected biomarkers}

To evaluate the quality of the SNPs selected by the different methods, we used
the performance of a classifier trained exclusively on those SNPs. The
underlying assumption is that SNPs that are good biomarkers will be good
predictiors of the phenotype of unseen samples. To that end, we followed a
5-fold cross-validation schema, where over 5 iterations 80\% of the data is used
to select SNPs and train the classifier, and 20\% is used to compute and
evaluate the predictions.

Specifically we fitted an L1-penalized logistic regression over the age and the selected SNPs, using the outcome (case/control) as
target variable. When the method retrieved a list of genes (all of them except
SConES), all the SNPs mapped to any of those genes were used.

\subsubsection{Comparison to external dataset}

\subsection{Code availability}

The implementation of the different network-based methods to GWAS data is
available at \url{https://github.com/hclimente/gwas-tools}. The code necessary
to reproduce all the analyses in this article is available at \url{https://github.com/hclimente/genewa}.

\bibliographystyle{plain}
\bibliography{bibliography}

\end{document}